\documentclass[utf]{uhb-inf}
\usepackage{pi3}


%% Daten der Uebungsgruppe (am besten in der Datei pi3.sty ablegen,
%% und mit \usepackage{pi3} immer wieder einbinden):

% Name des Tutors:
\tutor{Joe Totalle}
% Nummer der Übungsgruppe und Teilnehmer (mit '\\' getrennt):
\gruppe{1}{Alice \\ Bob \\ Claire \\ Dan}
% Nummer des Blattes und Abgabe-Datum
\zettel{1}{17.11.2008}

\begin{document}

\section{Ein erstes Haskell-Programm}

% Der \include-Befehl wurde so modifiziert, dass man auch andere
% Module einbinden kann. Hier ein Beispiel, welches das Modul
% "Aufgabe9" (in Datei "Aufgabe9.lhs") einbindet:
\include{Aufgabe9}
% Um die Zeilennummerierung nicht durcheinanderzubringen, sollte
% Haskell-Code nur in \include'ten Dateien enthalten sein.

\section{Literate Haskell}

% Hier wird ein weiteres Modul eingebunden:
\include{Aufgabe10}

\section{Ende gut, alles gut\ldots}

Wollen wir doch hoffen.

\end{document}

