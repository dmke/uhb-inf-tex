\documentclass[utf]{uhb-inf}

\veranstaltung{PI1}{Praktische Informatik 1}
\semester{WS2009/10}
\tutor{niemand}
\gruppe{dmke}{Dominik Menke}
\zettel{1}{04.11.2009}

\begin{document}
\section{Addieren\hfill 60\%}

Das Addieren soll dir den Einstieg in die Bearbeitung der Übungsaufgaben geben. Wärm dich an diesen Aufgaben auf, bevor es richtig zur Sache geht.

\subsection{}

Addiere diese kleinen Zahlen:

\[ 1 + 2 = \underline{\quad} \]

\subsection{}

Addiere diese größeren Zahlen

\[ 9 + 1 = \underline{\quad} \]

\section{Multiplizieren \hfill 40\%}

Jetzt wird es richtig kompliziert. Manchmal kann das Internet eine Hilfe sein, aber gib acht, dass du alle Quellen nennst!

\subsection{}

Multipliziere diese Zahlen miteinander:

\[ 1 * 3 = \underline{\quad} \]



\end{document}